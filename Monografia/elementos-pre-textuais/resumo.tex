Este trabalho tem como objetivo principal o desenvolvimento de um modelo para a análise de genomas virais, baseado no uso de códons. Essa ferramenta se propõe a ser uma importante ferramenta para a análise da evolução de espécies, utilizando sequências genômicas do SARS-COV-2 como base de estudo. A implementação desse modelo visa proporcionar maior eficiência computacional e alcançar resultados mais precisos. Adicionalmente, a ferramenta será capaz de apresentar visualizações gráficas dos resultados obtidos, facilitando a interpretação dos dados e auxiliando na tomada de decisões científicas. Espera-se que essa abordagem proporcione insights valiosos sobre a evolução de espécies virais, contribuindo para o avanço da virologia e da genômica comparativa. Os resultados obtidos serão analisados com o objetivo de demonstrar a eficácia dessa ferramenta na compreensão dos padrões evolutivos em espécies virais, tornando-a uma promissora aliada para pesquisadores e profissionais da área.

% Separe as palavras-chave por ponto
\palavraschave{Bioinformática; Códons; Filogenia; Viral.}