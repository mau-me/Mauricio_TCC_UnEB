%%%%%%%%%%%%%%%%%%%%%%%%%%%%%%%%%%%%%%%%%%%%%%%%%%%%%%%%%%%%%%%%%%%%%%%%%%%
%%%                         O MODELO                                    %%%
%%%%%%%%%%%%%%%%%%%%%%%%%%%%%%%%%%%%%%%%%%%%%%%%%%%%%%%%%%%%%%%%%%%%%%%%%%%

\chapter{Capítulo 3 -~Descrição do Projeto}

% objetivos, ponto de partida, produtos e resultados esperados, tecnologias a serem usadas.
% + estrutura da solução nas fases de treinamento e aplicação

% Lista de Tecnologias/Ferramentas utilizadas:
% Python 3
% Selenium
% Biopython
% Shell (verificar necessidade)
% Jupyter Notebook
% Minimap2
% GoFasta
% virtualEnv
% AGUA (verificar e apresentar os sub-softwares utilizados, ex.: CLOPE)


% ### METODOLOGIA ###

% ### Falar das seguintes ferramentas:
%   - Trello
%   - Github

A seguir, serão apresentadas a metodologia adotada neste estudo, bem como as etapas detalhadas de sua implementação.

% Metodologia
Um ponto importante para a obtenção dos objetivos deste trabalho está relacionada a definição da metodologia que servirá como alicerce. Com a proposta de desenvolver e validar um método de análise da evolução molecular de vírus com base no uso de códons, a metodologia escolhida para isso é o \gls{dsr}. Essa metodologia, proporciona um framework teórico e prático para a criação de artefatos inovadores, como métodos, modelos ou frameworks, visando resolver problemas específicos~\cite{peffers_dsr_2007}. Neste projeto, a ferramenta de análise de genes virais baseada em códons é o artefato que será desenvolvido e avaliado. Além disso, o \gls{dsr} enfatiza a validação e a avaliação da utilidade e eficácia do artefato em relação aos seus objetivos práticos. No caso deste projeto, a validação será realizada através da comparação dos resultados obtidos com a ferramenta proposta em relação às técnicas clássicas filogenéticas, que são amplamente utilizadas para a análise de genes virais. Essa comparação permitirá avaliar a eficácia e o valor agregado da abordagem baseada em códons.

Para a obtenção de sucesso ao utilizar o \gls{dsr} os seguintes passos serão seguidos:
\begin{enumerate}
  \item Identificação do problema e definição dos objetivos.
  \item Desenvolvimento dos artefatos.
  \item Avaliação do artefato.
  \item Apresentar contribuições científicas.
\end{enumerate}

Também será utilizada analises quantitativas, ou seja, medidas estatísticas para mensurar e comparar os resultados obtidos.\\
A pesquisa quantitativa só tem sentido quando há um problema muito bem definido e há informação e teoria a respeito do objeto de conhecimento, entendido aqui como o foco da pesquisa e/ou aquilo que se quer estudar. Esclarecendo mais, só se faz pesquisa de natureza quantitativa quando se conhece as qualidades e se tem controle do que se vai pesquisar.\cite{da_silva_pesquisa_2014}

\section{Plano de Implementação}

Os pontos a seguir serão realizados durante o desenvolvimento do projeto:

\begin{itemize}
  \item Coleta de sequências que serão utilizadas.
  \item Tratamento necessário dos dados.
  \item Avaliação de desempenho do modelo.
  \item Analise comparativa com modelos convencionais.
  \item Disponibilização do modelo como ferramenta web.
\end{itemize}

Com base nos itens apresentados anteriormente existem três grandes pilares a serem executados: Montagem e preparação do dataset a ser utilizado pelo modelo; Desenvolvimento completo do modelo, com todos as definições, implementações, testes e correções necessárias; e a análise comparativa que será realizada com um outro método existente e já tradicional. Esses pontos são apresentados de forma minuciosa a seguir:

\begin{itemize}
  \item \textbf{Montagem e Preparação do Dataset:} O download das sequências necessárias para a realização do projeto será realizada através do site \gls{bvbrc}. O \gls{bvbrc}, é um centro de recursos de bioinformática dedicado ao estudo e análise de bactérias e vírus. O site também disponibiliza uma uma coleção abrangente de banco de dados, incluindo sequências genômicas, anotações funcionais, informações de expressão gênica e estruturas tridimensionais. O acesso aos bancos de dados é dado por meio de uma interface amigável, onde é possível realizar pesquisas avançadas.
        Abaixo, está apresentado o passo a passo que será seguido nesta etapa:
        \begin{itemize}
          \item Analise do site \gls{bvbrc}.
          \item Desenvolvimento do script para download automático das sequências.
          \item Implementar procedimento de filtragem de sequências, removendo redundâncias e mantendo apenas sequências únicas.
          \item Implementar procedimento de extração do gene de interesse das sequências obtidas anteriormente.
          \item Utilizar um algoritmo de alinhamento múltiplo para alinhar as sequências obtidas.
          \item Desenvolver algoritmo para traduzir as sequências de \gls{dna} em sequências de códons.
        \end{itemize}

  \item \textbf{Desenvolvimento do Modelo:} Para a realização do desenvolvimento do modelo serão realizados os seguintes passos:
        \begin{itemize}
          \item Levantamento dos requisitos.
          \item Definir a arquitetura e a abordagem do modelo de classificação baseado em códons.
          \item Implementar o modelo utilizando uma biblioteca ou framework adequado.
          \item Realizar treinamento do modelo utilizando os dados preparados.
          \item Avaliar o desempenho do modelo utilizando métricas apropriadas.
                % \item \item Avaliar o desempenho do modelo utilizando métricas apropriadas, como acurácia, precisão e recall.
          \item Identificar possíveis problemas e realizar ajustes no modelo.
                % \item Identificar possíveis problemas de overfitting ou underfitting e realizar ajustes no modelo, como ajuste de hiperparâmetros ou utilização de técnicas de regularização.
        \end{itemize}

  \item \textbf{Análise comparativa entre o método proposto e outro método existente:} Será realizada uma analise com um conjunto de dados, onde será realizado analises estatísticas para verificação de melhorias, ou não, do novo método proposto analisando os seguintes aspectos:
        \begin{itemize}
          \item Comparação dos métodos de agrupamento adotados, avaliando sua eficácia na formação de clusters e na identificação de padrões ou similaridades nas sequências.
          \item Avaliação do custo computacional (tempo de execução e recursos requeridos) para a classificação das sequências em cada método.
          \item Comparação da eficiência computacional entre os métodos, considerando a escalabilidade e o desempenho em grandes volumes de dados.
        \end{itemize}
\end{itemize}

% Depois dos capítulos de fundamentação teórica e trabalhos relacionados, é hora de apresentar o trabalho propriamente dito. Descrever o que foi feito para desenvolver a solução do problema investigado. Aqui a metodologia deve ser bem detalhada, a arquitetura das soluções descritas em detalhes, as tecnologias utilizadas apresentadas e justificadas. Enfim, aqui é uma espécie de relatório do que você fez.

% Mais uma vez, alguns autores preferem neste mesmo capítulo apresentar também o plano de testes e validação e os resultados obtidos. Outros autores colocam isto num capítulo separado depois deste. Novamente não há regra rígida para isto. As duas opções são válidas e devem ser decididas em conjunto com o orientador. Sempre usar o bom senso para não ter capítulos curtos demais nem longos demais. É importante deixar muito claro o plano de validação, a fundamentação estatística utilizada, o processo de coleta de dados, e apresentar o resultado de forma adequada usando gráficos, tabelas, etc. Estes resultados devem ser analisados, mas NÃO DEVEM GERAR CONCLUSÕES AINDA. Analisar não significa concluir. Para isto existe um capítulo de Conclusão!


% \section{Plano de Testes e Validação}

% \section{Resultados Obtidos}