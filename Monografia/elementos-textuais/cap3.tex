%%%%%%%%%%%%%%%%%%%%%%%%%%%%%%%%%%%%%%%%%%%%%%%%%%%%%%%%%%%%%%%%%%%%%%%%%%%
%%%                         O MODELO                                    %%%
%%%%%%%%%%%%%%%%%%%%%%%%%%%%%%%%%%%%%%%%%%%%%%%%%%%%%%%%%%%%%%%%%%%%%%%%%%%

\chapter{Capítulo 3 - Descrição do Projeto}
\label{ch:projeto}

Depois dos capítulos de fundamentação teórica e trabalhos relacionados, é hora de apresentar o trabalho propriamente dito. Descrever o que foi feito para desenvolver a solução do problema investigado. Aqui a metodologia deve ser bem detalhada, a arquitetura das soluções descritas em detalhes, as tecnologias utilizadas apresentadas e justificadas. Enfim, aqui é uma espécie de relatório do que você fez. 

Mais uma vez, alguns autores preferem neste mesmo capítulo apresentar também o plano de testes e validação e os resultados obtidos. Outros autores colocam isto num capítulo separado depois deste. Novamente não há regra rígida para isto. As duas opções são válidas e devem ser decididas em conjunto com o orientador. Sempre usar o bom senso para não ter capítulos curtos demais nem longos demais. É importante deixar muito claro o plano de validação, a fundamentação estatística utilizada, o processo de coleta de dados, e apresentar o resultado de forma adequada usando gráficos, tabelas, etc. Estes resultados devem ser analisados, mas NÃO DEVEM GERAR CONCLUSÕES AINDA. Analisar não significa Concluir. Para isto existe um capítulo de Conclusão ! 