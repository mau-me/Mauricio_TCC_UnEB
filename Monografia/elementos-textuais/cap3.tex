%%%%%%%%%%%%%%%%%%%%%%%%%%%%%%%%%%%%%%%%%%%%%%%%%%%%%%%%%%%%%%%%%%%%%%%%%%%
%%%                         O MODELO                                    %%%
%%%%%%%%%%%%%%%%%%%%%%%%%%%%%%%%%%%%%%%%%%%%%%%%%%%%%%%%%%%%%%%%%%%%%%%%%%%

\chapter{Capítulo 3 - Descrição do Projeto}

A seguir, serão apresentadas a metodologia adotada neste estudo, bem como as etapas detalhadas de sua implementação.

\section{Plano de Implementação}

\begin{itemize}
  \item \textbf{Desenvolvimento do Modelo:} Para a realização do desenvolvimento do modelo serão realizados os seguintes passos:
        \begin{itemize}
          \item Definição do problema a ser solucionado.
          \item Levantamento dos requisitos.
          \item Coleta dos dados relevantes para a implementação.
          \item Análise exploratória dos dados.
          \item Pré-processamento dos dados obtidos.
          \item Seleção e engenharia de recursos.
          \item Escolha do algoritmo.
          \item Treinamento do modelo.
          \item Validação e avaliação.
          \item Ajustes e otimização.
          \item Teste do modelo.
        \end{itemize}

  \item \textbf{Montagem do Dataset:} O download das sequências necessárias para a realização do projeto será realizada através do site \gls{bvbrc}. O \gls{bvbrc}, é um centro de recursos de bioinformática dedicado ao estudo e análise de bactérias e vírus. O site também disponibiliza uma uma coleção abrangente de banco de dados, incluindo sequências genômicas, anotações funcionais, informações de expressão gênica e estruturas tridimensionais. O acesso aos bancos de dados é dado por meio de uma interface amigável, onde é possível realizar pesquisas avançadas.
        Também será realizado os tratamentos necessários para que os dados obtidos estejam padronizados no formato esperado pelo modelo a ser desenvolvido.

  \item \textbf{Análise comparativa entre o método proposto e outro método existente:} Será realizada uma analise com um conjunto de dados, onde será realizado analises estatísticas para verificação de melhorias, ou não, do novo método proposto.
\end{itemize}

% Depois dos capítulos de fundamentação teórica e trabalhos relacionados, é hora de apresentar o trabalho propriamente dito. Descrever o que foi feito para desenvolver a solução do problema investigado. Aqui a metodologia deve ser bem detalhada, a arquitetura das soluções descritas em detalhes, as tecnologias utilizadas apresentadas e justificadas. Enfim, aqui é uma espécie de relatório do que você fez.

% Mais uma vez, alguns autores preferem neste mesmo capítulo apresentar também o plano de testes e validação e os resultados obtidos. Outros autores colocam isto num capítulo separado depois deste. Novamente não há regra rígida para isto. As duas opções são válidas e devem ser decididas em conjunto com o orientador. Sempre usar o bom senso para não ter capítulos curtos demais nem longos demais. É importante deixar muito claro o plano de validação, a fundamentação estatística utilizada, o processo de coleta de dados, e apresentar o resultado de forma adequada usando gráficos, tabelas, etc. Estes resultados devem ser analisados, mas NÃO DEVEM GERAR CONCLUSÕES AINDA. Analisar não significa concluir. Para isto existe um capítulo de Conclusão!


% \section{Plano de Testes e Validação}

% \section{Resultados Obtidos}