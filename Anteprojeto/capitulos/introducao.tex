%%%%%%%%%%%%%%%%%%%%%%%%%%%%%%%%%%%%%%%%%%%%%%%%%%%%%%%%%%%%%%%%%%%%%%%%%%%
%%%                         INTRODUÇÃO                                  %%%
%%%%%%%%%%%%%%%%%%%%%%%%%%%%%%%%%%%%%%%%%%%%%%%%%%%%%%%%%%%%%%%%%%%%%%%%%%%

\chapter{Introdução}

\setlength{\parskip}{0.3cm}
%\thispagestyle{headings}


%A Introdução deve incluir a contextualização do tema como já realizado no anteprojeto, revisada e melhorada. Também dever caracterizar claramente o problema de pesquisa, a justificativa/motivação/relevância para a investigação realizada, os objetivos gerais e específicos. Para ajudar na justificativa, devem ser feita uma sintética revisão dos trabalhos relacionados apontando especialmente as lacunas que serão exploradas pelo presente trabalho. Alguns autores podem optar por já realizar uma completa revisão de literatura (trabalhos relacionados) na introdução enquanto outros fazem isto sucintamente na introdução deixando para aprofundar em um capítulo específico de Trabalhos Relacionados ou Revisão de Literatura. Discuta com seu orientador qual a melhor abordagem para o seu trabalho.

Os desafios impostos pela pandemia do COVID-19 incluíram a falta de conhecimento suficiente para a compreensão da importância das ameaças biológicas e para a preparação médica, apesar dos avanços científicos e tecnológicos. O conhecimento prévio sobre os agentes biológicos com potencial para causar pandemias pode melhorar substancialmente nossa preparação pré-pandemia.~\cite[p. 1]{behl_threat_2022}

Diante disso, a bioinformática, que é a junção de métodos computacionais e técnicas estatísticas com o objetivo de extrair informações de dados biológicos brutos, desempenha um papel fundamental na interpretação de dados genômicos e na compreensão de processos evolutivos.
Segundo~\cite[p.1]{barry_phylogenetic_analysis_2006} ``os métodos filogenéticos podem ser usados para analisar os dados da sequência de nucleotídeos de forma que a ordem de descendência de cepas relacionadas possa ser determinada. Quando associada à análise filogenética apropriada, a epidemiologia molecular tem o potencial de elucidar os mecanismos que levam a surtos microbianos e epidemias.''

A reconstrução filogenética é uma das abordagens amplamente utilizadas na análise da evolução de espécies, que permite investigar as relações evolutivas entre diferentes linhagens de vírus. Essas observações são realizadas com base em dados como sequências de \ac{dna} e \ac{rna}. Essas sequências são formadas por blocos fundamentais chamados de nucleotídeos, que são compostos por uma base nitrogenada, um açúcar e um grupo fosfato. As bases presentes nos nucleotídeos do \ac{dna} são adenina (A), timina (T), citosina (C) e guanina (G), enquanto no \ac{rna} a base timina é substituída pela uracila (U).~\cite{alberts2002molecular}

Uma das principais formas de análise filogenética é realizada através da árvore filogenética, onde são representadas as relações evolutivas entre um conjunto de espécies. De acordo com~\cite{morrison_tree_thinking} elas tem função importante porque apresentam de forma sucinta e particular a evolução dos descendentes partindo de ancestrais em comum.

A semelhança genética entre vários vírus infecciosos e mortais fornece uma visão do fato de que o \ac{rna} é a chave para discernir e marcar os possíveis patógenos que podem causar uma pandemia. Embora um padrão geral e motivos conservados possam ser observados em ancestrais imediatos, as regiões não conservadas das sequências são o resultado da acumulação de mutações, seja por inserção ou deleção de um ou vários nucleotídeos ou por substituição pontual de um nucleotídeo por outro. A fonte principal de mutações em vírus são percalços na replicação e a recombinação de \ac{rna}~\cite[p. 11]{behl_threat_2022}.

% Nos eucariotos, a principal fonte de mutação é a radiação incidente nas gônadas, que pode ser ambiental (solo com material radioativo), artificial (radiografias e tomografias) ou radiação cósmica.

Apesar da utilidade da filogenética e dos softwares comerciais e públicos disponíveis para análises filogenéticas, os métodos filogenéticos são muitas vezes aplicados de forma inadequada. Mesmo quando aplicados adequadamente, são mal explicados e, portanto, mal compreendidos.~\cite[p. 1]{barry_phylogenetic_analysis_2006} Além disso, por trabalhar com grandes quantidades de dados, os métodos utilizados devem ser avaliados também em relação ao seu custo computacional.

Na busca de trabalhos relacionados, vários métodos foram encontrados, e a seguir são apresentados.

O método de \ac{ml} (ou \textit{Maximum Likelihood}), não é exclusivo da filogenia, mas sim uma abordagem estatística. A sua aplicação em filogenia consiste em avaliar a probabilidade de que o modelo de evolução escolhido gere os dados observados, que são por exemplo, características de um organismo. Essa proposta foi utilizada por~\cite{fall_genetic_diversity_2021,behl_threat_2022,shabbir_comprehensive_2020,hudu_hepatitis_2018,sallard_tracing_2021,paez-espino_diversity_evolution_2019,tang_evolutionary_2021,cho_analysis_2022}.

Já em~\cite{yin_systematic_2019, bedoya-pilozo_molecular_epidemiology_2018}, foi usada a inferência bayesiana, que é fundamentada no teorema de Bayes, que permite a atualização das probabilidades a priori para probabilidades a posteriori à medida que novas evidências são incorporadas.

Além desses,~\cite{potdar_phylogenetic_2021} utilizou a junção de vizinhos (ou \textit{Neighbor-Joining}), que é baseado em uma abordagem heurística que visa construir uma árvore filogenética a partir de uma matriz de distância entre as sequências estudadas. O trabalho de~\cite{lichtblau_alignment-free_2019} expos o Frequency Chaos Game Representation e~\cite{kim_ngs_2022} a floresta aleatória. Por fim,~\cite{dimitrov_updated_2019} comparou três modelos para reconstrução de árvores filogenéticas: junção de vizinhos; \ac{ml} e inferência bayesiana.

As soluções até então desenvolvidas, são guiadas pela reconstrução das árvores filogenéticas construídas a partir das mutações de nucleotídeos. Neste aspecto, as ferramentas disponíveis não oferecem uma aplicação no contexto de árvores reconstruídas com distâncias obtidas a partir da diferença do uso de códons, que são sequências de três nucleotídeos responsáveis pela codificação dos aminoácidos nas proteínas. Os códons desempenham um papel crucial na determinação da função e estrutura das proteínas, e alterações nos códons podem resultar em mudanças significativas nas características fenotípicas dos vírus. Necessita-se então, de pesquisas e desenvolvimento de ferramentas que realizem uma classificação de sequências genéticas com base no uso/frequência de códons.

% Checar se é necessário colocar...
% O método de junção de vizinhos é especialmente útil quando o número de sequências a serem analisadas é da ordem de centenas ou milhares. Além disso, a precisão das árvores por ele geradas é semelhante a outros métodos mais demorados para conjuntos de dados relativamente pequenos (200 sequências). O método constrói árvores agrupando sequências vizinhas de maneira gradual. Em cada etapa do agrupamento de sequências, ele minimiza a soma dos comprimentos dos ramos e, assim, examina múltiplas topologias. No entanto, para grandes conjuntos de dados, NJ examina apenas uma fração minúscula do número total de topologias possíveis.~\cite[p. 1]{tamura_prospects_2004}

% Comentários
% Escrever aqui a contextualização do tema. Falar sobre a grande área em que o problema investigado está inserido. Este capítulo deve apresentar o referencial teórico necessário para o entendimento do tema, do problema,  dos conceitos e tecnologias envolvidos e do que já foi realizado nos trabalhos relacionados (fruto da revisão sistemática realizada) e das lacunas identificadas (o que não foi feito), desenvolvendo uma ligação entre o contexto e o problema de pesquisa.
% Normalmente autores de referência são usados nesta contextualização.

% A Introdução deve ser finalizada definindo claramente o problema de pesquisa. O problema deve ser contextualizado e descrito detalhadamente. Não confundir, problema com solução. A solução proposta (hipótese) irá aparecer nos objetivos.

% Espera-se um capítulo de aproximadamente 3 páginas aqui.