\documentclass[font=plain]{abnt}

\usepackage[utf8]{inputenc}
\usepackage[brazil]{babel}
\usepackage{abntex2cite}
\usepackage{graphicx}
\usepackage{graphicx,color}
\usepackage{paralist}
\usepackage{subfloat}
\usepackage{subfig}
\usepackage{multirow}
\usepackage{booktabs}



\graphicspath{{imagens/}}

\instituicao{UNEB - Universidade do Estado da Bahia
             \par Departamento de Ciências Exatas e da Terra
             \par Colegiado de Sistemas de Informação}
\titulo{Ferramenta Computacional para o Estudo da Evolução de Espécies Baseado no Uso de Códons}
\autor{Mauricio Souza Menezes}
\orientador{PhD Diego Gervasio Frias Suárez}
\coorientador{PhD Vagner Fonseca}
\comentario{\textbf{Áreas da Computação:}
           \\ Bioinformática}
\local{Salvador-BA}
%esta data deve ficar estática após entrega final
\data{\today}

\begin{document}
\capa
\folhaderosto

\begin{folhadeaprovacao}
    \begin{center}
        \large
        \textbf{Folha de Aprovação}
    \end{center}

    Anteprojeto sob o título provisório \textit{Ferramenta Computacional para o Estudo da Evolução de Espécies Baseado no Uso de Códons} apresentado como exigência parcial para avaliação na disciplina Trabalho de Conclusão de Curso I do bacharelado em Sistemas de Informação da Universidade do Estado da Bahia entregue por \textit{Mauricio Souza Menezes} a Marco Antônio Costa Simões professor da disciplina, em \today, em Salvador, Bahia.
    \setlength{\ABNTsignthickness}{0.4pt}
    \setlength{\ABNTsignskip}{2cm}
    \hspace*{1cm}
    \assinatura{Mauricio Souza Menezes\\Orientando}
    \hspace*{1cm}
    \assinatura{Diego Gervasio Frias Suárez}
\end{folhadeaprovacao}

%\begin{resumo}

%\end{resumo}


\sumario


\chapter{Introdução}

Os desafios impostos pela pandemia do COVID-19 incluíram a falta de conhecimento suficiente na compreensão da importância das ameaças biológicas e da preparação médica, apesar dos avanços científicos e tecnológicos. O conhecimento prévio sobre os agentes biológicos com potencial para causar pandemia pode melhorar substancialmente nossa preparação pré-pandemia.~\cite[p. 1]{behl_threat_2022}

Segundo~\cite[p.1]{barry_phylogenetic_analysis_2006} `os métodos filogenéticos podem ser usados para analisar os dados da sequência de nucleotídeos de forma que a ordem de descendência de cepas relacionadas possa ser determinada. Quando associada à análise filogenética apropriada, a epidemiologia molecular tem o potencial de elucidar os mecanismos que levam a surtos microbianos e epidemias.'

Uma das principais formas de análise filogenética é realizada através da árvore filogenética, onde são representadas as relações evolutivas entre um conjunto de espécies. De acordo com~\cite{morrison_tree_thinking} elas tem função importante porque apresentam de forma sucinta e particular a evolução dos descendentes partindo de ancestrais em comum.

A partir da análise filogenética realizada, pode-se \underline{facilmente} [X?] interpretar que algumas cadeias de sequências de nucleotídeos podem resultar em \underline{catástrofe} [X?]. A semelhança genética entre vários vírus infecciosos e mortais fornece uma visão do fato de que o RNA é a chave para discernir e marcar os possíveis patógenos que podem causar uma pandemia. Embora um padrão geral e motivos conservados possam ser observados em ancestrais imediatos, \underline{rescrito[} as regiões não conservadas das sequências são o resultado da acumulação de mutações, seja por inserção ou deleção de um ou vários nucleotídeos ou por substituição pontual de um nucleotídeo por outro. A fonte principal de mutações em vírus são percalços na replicação e a recombinação de RNA ~\cite[p. 11]{behl_threat_2022} \underline{]rescrito}. Nos eucariotos, a principal fonte de mutação é a radiação incidente nas gônadas, que pode ser ambiental (solo com material radioativo), artificial (radiografias e tomografias) ou radiação cósmica.

Apesar da utilidade da filogenética e dos softwares \underline{rescrito[} comerciais e públicos \underline{]rescrito} disponíveis para análises filogenéticas, os métodos filogenéticos são muitas vezes aplicados de forma inadequada. Mesmo quando aplicados adequadamente, muitas vezes são mal explicados e, portanto, mal compreendidos.~\cite[p. 1]{barry_phylogenetic_analysis_2006} Além disso, por trabalhar com grandes quantidades de dados, os métodos utilizados devem ser avaliados também em relação ao seu custo computacional.

% AINDA FALTA COMPLEMENTAR
% CITAR TRABALHOS SOBRE USO DE CÓDONS
% Falta Falar sobre o CBUC

% Comentários
% Escrever aqui a contextualização do tema. Falar sobre a grande área em que o problema investigado está inserido.  Este capítulo deve apresentar o referencial teórico necessário para o entendimento do tema, do problema,  dos conceitos e tecnologias envolvidos e do que já foi realizado nos trabalhos relacionados (fruto da revisão sistemática realizada) e das lacunas identificadas (o que não foi feito), desenvolvendo uma ligação entre o contexto e o problema de pesquisa.
% Normalmente autores de referência são usados nesta contextualização.

% A Introdução deve ser finalizada definindo claramente o problema de pesquisa. O problema deve ser contextualizado e descrito detalhadamente. Não confundir, problema com solução. A solução proposta (hipótese) irá aparecer nos objetivos.

% Espera-se um capítulo de aproximadamente 3 páginas aqui.

\chapter{Objetivos}
Com base no problema de pesquisa apresentado na seção anterior apresentamos foi definido como objetivo geral o desenvolvimento de um método para a criação de cladogramas a partir da diferença do uso de códons. Além desse, os objetivos específicos são apresentados a seguir:
\begin{itemize}
    \item Definir um padrão comparativo entre o modelo proposto e árvores construídas a partir das mutações de nucleotídeos.
\end{itemize}

% Não funcionou assim...
% \subsection{Objetivos Gerais}
% Desenvolver um método capaz de realizar a análise filogenética com base no uso/frequência de códons.
% \subsection{Objetivos Específicos}
% Desenvolver um método capaz de realizar a análise filogenética com base no uso/frequência de códons.

% Escrever de forma clara, o Objetivo Geral do Trabalho (a meta que se pretende atingir com o final do trabalho, o que se pretende fazer) e os Objetivos Específicos (Objetivos menores, parciais que precisam ser atingidos como parte do desenvolvimento do Trabalho OU Objetivos Secundários que complementam o objetivo Geral ou são consequências do mesmo). Cuidado ao escrever os objetivos, além de claros eles precisam ser exequíveis. Os objetivos são as soluções hipotéticas apresentadas para solucionar o problema descrito na introdução.

\chapter{Justificativas e Contribuições}
Escrever a Motivação (por que) para a execução deste trabalho, destacando a relevância (social, econômica ou acadêmica) do mesmo. Descrever as razões pelas quais o projeto deve ser desenvolvido, quais as contribuições (acadêmicas e/ou científicas) para a área de conhecimento do projeto. Tal contribuição é assegurada pela utilidade do trabalho aos demais, pela contribuição cumulativa (ou seja, pelo que este acrescenta ao conjunto do conhecimento científico do tema), pelo ineditismo do tema ou da abordagem e pela contribuição à superação de lacunas no conhecimento.

\chapter{Metodologia}
Escrever detalhadamente a Metodologia (como será feito)  que será usada para execução do Trabalho. Em outras palavras, dizer como os objetivos geral e específicos serão atingidos, que ações serão executadas, que testes serão realizados, que indicadores serão usados, etc. Descrever também as tecnologias que serão utilizadas para o desenvolvimento do trabalho e como se pretende validar o projeto. Lembre de usar a fundamentação em estatística para trabalhos que tenham isto como requisito (pesquisas de campo, aplicação de questionários, estudos de caso, etc). Ao descrever as ações na metodologia não economize em detalhes. Quanto mais detalhadas forem as ações, mais fácil será estimar prazos exequíveis no cronograma.


\chapter{Cronograma}

Descrever o cronograma de execução do trabalho planejando a duração de cada uma das etapas previstas na metodologia até a conclusão do mesmo ao final da disciplina TCC-02.

O cronograma deste projeto é apresentado nas tabelas \ref{tab:cronograma-1-2} e \ref{tab:cronograma-2-2}.

\begin{table}[ht]
    \centering
    \begin{tabular}{ p{7.8cm} c c c c c c c }
        \toprule
         & Jan       & Fev       & Mar       & Abr       & Mai       & Jun       & \\
        \midrule
        Atividade 1
         & $\bullet$ & $\bullet$ & $\bullet$ &           &           &           & \\
        \midrule
        Atividade 2
         &           & $\bullet$ & $\bullet$ &           &           &           & \\
        \midrule
        Atividade 3
         &           & $\bullet$ & $\bullet$ & $\bullet$ &           &           & \\
        \midrule
        Atividade 4
         &           &           & $\bullet$ &           &           &           & \\
        \midrule
        Atividade 5
         &           &           &           & $\bullet$ & $\bullet$ & $\bullet$ & \\
        \bottomrule
    \end{tabular}
    \caption{Cronograma de Janeiro a Junho de 20XX}
    \label{tab:cronograma-1-2}
\end{table}

\begin{table}[hb]
    \centering
    \begin{tabular}{ p{7.8cm} c c c c c c c }
        \toprule
         & Jul       & Ago       & Set       & Out       & Nov       & Dez       & \\
        \midrule
        Atividade 6
         & $\bullet$ & $\bullet$ &           &           &           &           & \\
        \midrule
        Atividade 7
         &           & $\bullet$ & $\bullet$ &           &           &           & \\
        \midrule
        Atividade 8
         &           & $\bullet$ & $\bullet$ & $\bullet$ &           &           & \\
        \midrule
        Atividade 9
         &           &           & $\bullet$ & $\bullet$ &           &           & \\
        \midrule
        Atividade 10
         &           &           &           & $\bullet$ & $\bullet$ &           & \\
        \midrule
        Atividade 11
         &           &           &           &           &           & $\bullet$ & \\
        \bottomrule
    \end{tabular}
    \caption{Cronograma de Julho até Dezembro de 20XX}
    \label{tab:cronograma-2-2}
\end{table}

\textcolor{red}{Observação: As tabelas acima são apenas ilustrativas, o aluno deve elaborar seu cronograma e apresentá-lo da forma mais adequada ao seu projeto, inclusive podendo usar uma única tabela, dividir os períodos em semanas, dias, etc.}

Incluir as Referências Bibliográficas pesquisadas durante a elaboração do Anteprojeto e que foram CITADAS no mesmo, já utilizando a norma da ABNT como nos exemplos\cite{exemplo1} \cite{exemplo2} \cite{exemplo3}.

\begin{citacao} Exemplo de citação direta com mais de três linhas.Exemplo de citação direta com mais de três linhas. Exemplo de citação direta com mais de três linhas. Exemplo de citação direta com mais de três linhas. Exemplo de citação direta com mais de três linhas. Exemplo de citação direta com mais de três linhas. Exemplo de citação direta com mais de três linhas\cite[p. 3]{exemplo1}.
\end{citacao}
Numa citação direta com menos de três linhas, ``coloca-se entre aspas e cita indicando a página" \cite[p. 2]{exemplo3}.

\bibliographystyle{abnt-num}
\bibliography{refGeral}

\end{document}